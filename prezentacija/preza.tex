%%%%%%%%%%%%%%%%%%%%%%%%%%%%%%%%%%%%%%%%%
% Beamer Presentation
% LaTeX Template
% Version 1.0 (10/11/12)
%
% This template has been downloaded from:
% http://www.LaTeXTemplates.com
%
% License:
% CC BY-NC-SA 3.0 (http://creativecommons.org/licenses/by-nc-sa/3.0/)
%
%%%%%%%%%%%%%%%%%%%%%%%%%%%%%%%%%%%%%%%%%

%----------------------------------------------------------------------------------------
%	PACKAGES AND THEMES
%----------------------------------------------------------------------------------------

\documentclass[utf8]{beamer}
\usepackage[T1]{fontenc}
\usepackage{amsmath}
\usepackage{amssymb}

\mode<presentation> {

% The Beamer class comes with a number of default slide themes
% which change the colors and layouts of slides. Below this is a list
% of all the themes, uncomment each in turn to see what they look like.

%\usetheme{default}
%\usetheme{AnnArbor}
%\usetheme{Antibes}
%\usetheme{Bergen}
%\usetheme{Berkeley}
%\usetheme{Berlin}
%\usetheme{Boadilla}
%\usetheme{CambridgeUS}
%\usetheme{Copenhagen}
%\usetheme{Darmstadt}
%\usetheme{Dresden}
%\usetheme{Frankfurt}
%\usetheme{Goettingen}
%\usetheme{Hannover}
%\usetheme{Ilmenau}
%\usetheme{JuanLesPins}
%\usetheme{Luebeck}
\usetheme{Madrid}
%\usetheme{Malmoe}
%\usetheme{Marburg}
%\usetheme{Montpellier}
%\usetheme{PaloAlto}
%\usetheme{Pittsburgh}
%\usetheme{Rochester}
%\usetheme{Singapore}
%\usetheme{Szeged}
%\usetheme{Warsaw}

% As well as themes, the Beamer class has a number of color themes
% for any slide theme. Uncomment each of these in turn to see how it
% changes the colors of your current slide theme.

%\usecolortheme{albatross}
%\usecolortheme{beaver}
%\usecolortheme{beetle}
%\usecolortheme{crane}
%\usecolortheme{dolphin}
%\usecolortheme{dove}
%\usecolortheme{fly}
%\usecolortheme{lily}
%\usecolortheme{orchid}
%\usecolortheme{rose}
%\usecolortheme{seagull}
%\usecolortheme{seahorse}
%\usecolortheme{whale}
%\usecolortheme{wolverine}

%\setbeamertemplate{footline} % To remove the footer line in all slides uncomment this line
%\setbeamertemplate{footline}[page number] % To replace the footer line in all slides with a simple slide count uncomment this line

%\setbeamertemplate{navigation symbols}{} % To remove the navigation symbols from the bottom of all slides uncomment this line
}

%\usepackage{graphicx} % Allows including images
\usepackage{booktabs} % Allows the use of \toprule, \midrule and \bottomrule in tables

%----------------------------------------------------------------------------------------
%	TITLE PAGE
%----------------------------------------------------------------------------------------

\title[Author profiling]{Author profiling} % The short title appears at the bottom of every slide, the full title is only on the title page

\author[Mrčela, Ratković, Žužul]{Lovre Mrčela, Marko Ratković, Ante Žužul} % Your name
\institute[FER] % Your institution as it will appear on the bottom of every slide, may be shorthand to save space
{
FAKULTET ELEKTROTEHNIKE I RAČUNARSTVA \\ % Your institution for the title page
\medskip
\textit{\{lovre.mrcela, marko.ratkovic, ante.zuzul\}@fer.hr} % Your email address
}
\date{\today} % Date, can be changed to a custom date

\begin{document}

\begin{frame}
\titlepage % Print the title page as the first slide
\end{frame}

\begin{frame}
\frametitle{Content} % Table of contents slide, comment this block out to remove it
\tableofcontents % Throughout your presentation, if you choose to use \section{} and \subsection{} commands, these will automatically be printed on this slide as an overview of your presentation
\end{frame}

%----------------------------------------------------------------------------------------
%	PRESENTATION SLIDES
%----------------------------------------------------------------------------------------

%------------------------------------------------
\section{Introduction} % Sections can be created in order to organize your presentation into discrete blocks, all sections and subsections are automatically printed in the table of contents as an overview of the talk
\frame{\tableofcontents[currentsection]}
%------------------------------------------------

%\subsection{Subsection Example} % A subsection can be created just before a set of slides with a common theme to further break down your presentation into chunks


%------------------------------------------------

\begin{frame}
	\frametitle{Introduction}
	What is author profiling?
	\pause
	\begin{itemize}
		\item Author profiling is process of determining author age-group, gender and Big five personality traits.
		\pause
		\item For this project we are trying to profile author base on their tweeter post.
		\pause
		\item Process have two separate task:
		\begin{itemize}
			\pause
			\item[$ \bullet $] classification task for age-group and gender
			\pause
			\item[$ \bullet $] regression task for Big five personality traits
		\end{itemize} 
	\end{itemize}
\end{frame}

%------------------------------------------------

\begin{frame}
	\frametitle{Introduction}
	\begin{itemize}
		\item Dataset for task was taken for PAN competition.
		\item Dataset consist 4 language: English, Spanish, Italian and Dutch.
		\item Official test set isn't available due this year PAN competition.
	\end{itemize}
\end{frame}

%------------------------------------------------


\section{Scope of project}
\frame{\tableofcontents[currentsection]}

\begin{frame}
	Approach to solving problem:
	\begin{itemize}
		\item find optimal set of features
		\item find optimal model
	\end{itemize}
\end{frame}

\begin{frame}
	\frametitle{Preprocessing tweets}
	\begin{itemize}
		\item substituting \textit{url} with \textit{\textbf{URL}}
		\item substituting \textit{usernames} (referenced in replies) with \textit{\textbf{REPLY}}
		\item removing stop words from set of user tweets
		\item converting all tweets to lowercase
		\item removing repetitions of letter in word (e.g 'coooool' to 'cool')
	\end{itemize}
\end{frame}

\begin{frame}
	\frametitle{Set of features}
	Features used for this problem:
	\begin{itemize}
		\item tf-idf weighting scheme used on trigrams representation of preprocessed user tweets
		\item number of emoticons
		\item number of consecutive long repetitions of characters
		\item number of replies
		\item number of hashtags
		\item number of exclamation marks
		\item average length and standard deviation of posts
		\item average length and standard deviation of words
	\end{itemize}
\end{frame}

%------------------------------------------------

\begin{frame}
	\frametitle{Models}
	Model used for classification:
	\begin{itemize}
		\item Logistic Regression
		\item Naive Bayes Classifier
		\item Decision Tree Classifier
		\item Random Forest Classifier
		\item SVC (using \textit{rbf}, linear, poly and sigmoid kernels)
	\end{itemize}
\end{frame}


\begin{frame}
	\frametitle{Models}
	Model used for regression:
	\begin{itemize}
		\item Linear Regression
		\item Decision Tree Regressor
		\item Random Forest Regressor
		\item SVR (using \textit{rbf}, linear, poly and sigmoid kernels)
	\end{itemize}
\end{frame}

%------------------------------------------------
\section{Testing}
\frame{\tableofcontents[currentsection]}
%------------------------------------------------
\begin{frame}
	\frametitle{Testing}
	\begin{itemize}
		\item Training dataset was divided into subset for training (70\%) and for validation (30\%)
		\item Optimal model and hyperparameters were selected by using 10–fold cross–validation
		\item We have used baseline models because official test set wasn't available at this time
	\end{itemize}
\end{frame}
%------------------------------------------------
\section{Results}

\begin{frame}
	\frametitle{Overview of additional features for each age-group per language}
	Ne mogu staviti Table 1 
\end{frame}

\begin{frame}
	\frametitle{Overview of additional features values for gender per language}
	Table 2
\end{frame}

\begin{frame}
	\frametitle{Overview of results of age–group classification per language}
	table 3	
\end{frame}


\begin{frame}
	table 4	
\end{frame}

\begin{frame}
	table 5	
\end{frame}
%------------------------------------------------
\section{Conclusion}
\frame{\tableofcontents[currentsection]}

%------------------------------------------------

\begin{frame}
	\frametitle{Conclusion}
	\begin{itemize}
		\item We succeeded to obtain results similar to other published works
		\item Possible upgrade:
		\begin{itemize}
			\item Latent Semantic Analysis
		\end{itemize}
	\end{itemize}
\end{frame}

%------------------------------------------------



\begin{frame}
	\frametitle{The End}
	\Huge{\centerline{Tnx for listening!}}
\end{frame}

%------------------------------------------------

\begin{frame}
\Huge{\centerline{Questions??}}
\end{frame}

%----------------------------------------------------------------------------------------

\end{document}