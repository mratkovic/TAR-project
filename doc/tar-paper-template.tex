% Tenplate for TAR 2014
% (C) 2014 Jan Šnajder, Goran Glavaš
% KTLab, FER

\documentclass[10pt, a4paper]{article}

\usepackage{tar2014}

\usepackage[utf8]{inputenc}
\usepackage[pdftex]{graphicx}
\usepackage{booktabs}
\usepackage{amsmath}
\usepackage{amssymb}

\usepackage[activate={true,nocompatibility},final,tracking=true,kerning=true,spacing=true,factor=1100,stretch=10,shrink=10]{microtype}

\title{Author profiling}

%VAŽNO: Zakomentirajte sljedeću liniju kada šaljete rad na recenziju
\name{Lovre Mrčela, Marko Ratković, Ante Žužul} 

\address{
University of Zagreb, Faculty of Electrical Engineering and Computing\\
Unska 3, 10000 Zagreb, Croatia\\ 
\texttt{\{lovre.mrcela, marko.ratkovic, ante.zuzul\}@fer.hr}\\
}
          
         
\abstract{ 
%This document provides the instructions on formatting the TAR project paper in \LaTeX. This is where you write the abstract (i.e., summary) of the work you carried out. The abstract is a paragraph of text ranging between 70 and 150 words.
The goal of this project was to profile an author by analyzing a set of texts written by them, and then determining degree of each the Big Five personality traits.
In addition, gender and age--group for each author are derived as well.
The dataset was collected from twitter profiles, in English, Italian, Spanish and Dutch.
%\{\_\} and \{\_\} approaches/techniques were used to achieve the results.
Approach was based on \textit{tf-idf}, considering occurences of trigrams.
}

\begin{document}

\maketitleabstract

\section{Introduction}

%This section is the introduction to your paper. Introduction should not be too elaborate, that is what other sections are for (the Introduction should definitely not spill over to second page).
Author profiling deals with problem of describing someone's personality, by way of extracting information from their writing style.
Personality can be described using five traits (the so-called \textit{``Big Five personality traits''}), which are: extraversion, stability, agreeableness, conscientiousness and openness to experience.
Degrees of each trait range from -0.5 (indicating the total opposite) to 0.5 (indicating the exact match).

%This is the second paragraph of the introduction. Paragraphs are in \LaTeX separated by inserting an empty line in between them.  Avoid very large paragraphs (larger than half of the page height) but also avoid tiny paragraphs (e.g., one-sentence paragraphs).
Provided with degrees of the five traits, it is possible to determine author's gender and age--group, via means of classification based on a model trained on previously labelled data.
In this project, we used the linear SVC and Gaussian naive Bayes models for the classification into gender and age--group, and the linear regression with squared error measure for determining the degrees of personality traits.
The training set we used was a collection of twitter posts in English, Spanish, Italian and Dutch authors, ranging from around 35 authors in Dutch to 150 in English, each author's file containing about 100 posts.

\section{Approach}

This is the second section. In scientific papers this is usually (but not necessarily) the section in which related research is (briefly) described. 

\subsection{Text preprocessing}

For the rest of the process to be optimal, some sort of text preprocessing needs to be done on the raw input data.
The input data we use is given in \textit{xml} format, so the first step in preprocessing was to parse the actual sentences from the \textit{xml} structure.
When that is done, following steps are also applied:
\begin{itemize}
	\item \textit{urls} to other sites are substituted with an \textsc{url} tag,
	\item other user names (when referenced in replies) are substituted with a \textsc{reply} tag,
	\item all the text is converted to lower case,
	\item each three consecutive letters are grouped into trigrams, and
	\item weighted vector of trigrams is obtained by using \textit{tf--idf} weighting scheme.
\end{itemize}

\subsection{Gender and age--group classification}
%\label{sec:first}

For the gender and age--group classification subproblem, following approaches were considered:
\begin{itemize}
	\item logistic regression
	\item naive Bayes classifier
	\item decision tree classifier
	\item random forest classifier
	\item SVC (using \textit{rbf}, linear, poly-- or sigmoid kernels)
\end{itemize}
\noindent Results of using each approach are compared in the table \ref{tab:classifiers}. The best result was obtained by using SVC with linear kernel.

\subsection{Personality traits regression}

%This is the second subsection of the second section. Referencing the (sub)sections in text is performed as follows: ``in Section \ref{sec:first} we have shown \dots''.

\subsubsection{Sub-subsection example} 

This is a sub-subsection. If possible, it is better to avoid sub-subsections. 

\section{Results}
The results.

\section{Extent of the paper}

The paper should have at least. The paper should have a minimum of 3 and a maximum of 5 pages plus an additional page for references.

\section{Figures and tables}

\subsection{Figures}

Here is an example on how to include figures in the paper. Figures are included in \LaTeX code immediately \textit{after} the text in which these figures are referenced. Allow \LaTeX to place the figure where it believes is best (usually on top of the page of at the position where you would not place the figure). Figures are references as follows: ``Figure~\ref{fig:figure1} shows \dots''. Use tilda (\verb.~.) to prevent separation between the word ``Figure'' and its enumeration. 

\begin{figure}
\begin{center}
\includegraphics[width=\columnwidth]{tar1314}
\caption{This is the figure caption. Full sentences should be followed with a dot. The caption should be placed \textit{below} the figure. Caption should be short; details should be explained in the text.}
\label{fig:figure1}
\end{center}
\end{figure}

\subsection{Tables}

There are two types of tables: narrow tables that fit into one column and a wide table that spreads over both columns.

\subsubsection{Narrow tables}

%An example of the narrows table is the Table~\ref{tab:narrow-table}. Do not use vertical lines in tables -- vertical tables have no effect and they make tables visually less attractive.

\begin{table}
\caption{Comparison of results obtained by using different classifiers.}
\label{tab:classifiers}
\begin{center}
\begin{tabular}{ll}
\toprule
Heading1 & Heading2 \\
\midrule
One & First row text \\
Two   & Second row text \\
Three   & Third row text \\
      & Fourth row text \\
\bottomrule
\end{tabular}
\end{center}
\end{table}

\subsection{Wide tables}

Table~\ref{tab:wide-table} is an example of a wide table that spreads across both columns. The same can be done for wide figures that should spread across the whole width of the page. 

\begin{table*}
\caption{Wide-table caption}
\label{tab:wide-table}
\begin{center}
\begin{tabular}{llr}
\toprule
Heading1 & Heading2 & Heading3\\
\midrule
A & A very long text, longer that the width of a single column & $128$\\
B & A very long text, longer that the width of a single column & $3123$\\
C & A very long text, longer that the width of a single column & $-32$\\
\bottomrule
\end{tabular}
\end{center}
\end{table*}

\section{Math expressions and formulas}

Math expressions and formulas that appear within the sentence should be writen inside the so-called \emph{inline} math environment: $2+3$, $\sqrt{16}$, $h(x)=\mathbf{1}(\theta_1 x_1 + \theta_0>0)$. Larger expressions and formulas (e.g., equations) should be written in the so-called \emph{displayed} math environment:

\[
b^{(i)}_k = \begin{cases}
1 & \text{ako 
    $k = \text{argmin}_j \| \mathbf{x}^{(i)} - \mathbf{\mu}_j \|$}\\
0 & \text{inače}
\end{cases}
\]

Math expressions which you reference in the text should be written inside the \textit{equation} environment:

\begin{equation}\label{eq:kmeans-error}
J = \sum_{i=1}^N \sum_{k=1}^K 
b^{(i)}_k \| \mathbf{x}^{(i)} - \mathbf{\mu}_k \|^2
\end{equation}

Now you can reference equation \eqref{eq:kmeans-error}. If the paragraphs continues right after the formula

\begin{equation}
f(x) = x^2 + \varepsilon
\end{equation}

\noindent like this one does, then use the command \emph{noindent} after the equation to prevent the indentation of the row starting the paragraph. 

Multiletter words in the math environment should be written inside the command \emph{mathit}, otherwise \LaTeX will insert spacing between the letters to denote the multicplication of values denoted by symbols. For example, compare
$\mathit{Consistent}(h,\mathcal{D})$ and\\
$Consistent(h,\mathcal{D})$.

If you need a math symbol, but you don't know the command for it in \LaTeX, try
\emph{Detexify}.\footnote{\texttt{http://detexify.kirelabs.org/}}

\section{Referencing literature}

References to other publications should be written in brackets with the last name of the first author and the year of publication, e.g., \citep{chomsky-73}.  Multiple references are written in sequence, one after another, separated by semicolon and without whitespaces in between, e.g., \citep{chomsky-73,chave-64,feigl-58}. References are typically written at the end of the sentence and necessarily before the sentence punctuation.

If the publication is authored by more than author, only the name of the first author is written, after which abbreviation \emph{et al.}, meaning \emph{et alia}, i.e.,~and others is written as in \citep{johnson-etc}. If the publication is authored by only two authors, then the last names of both authors are written \citep{johnson-howells}.

If the name of the author is incorporated into the text of the sentence, it should be out of the brackets (only the year should be in the brackets). E.g.,~``\citet{chomsky-73}
suggested that \dots''. The difference is whether you reference the publication or the author who wrote it. 

The list of all literature references is given alphabetically at the end of the paper. The form of the reference depends on the type of the bibliographic unit: conference papers,
\citep{chave-64}, books \citep{butcher-81}, journal articles
\citep{howells-51}, doctoral dissertations \citep{croft-78} and book chapters \citep{feigl-58}. 

All of this is produced for you automatically by using BibTeX. Sve ovo dobivate automatski ako. In the file \texttt{tar2014.bib} insert the BibTeX entries, and then reference them via their symbolic names.

\section{Conclusion}

Conclusion is the last enumerated section of the paper. Conclusion should not exceed half of the column and is typically be split into 2--3 paragraphs.

\section*{Acknowledgements}

If suited, before inserting the literature references you can include the Acknowledgements section in order to thank those who helped you in any way to deliver the paper, but are not co-authors of the paper.

\bibliographystyle{tar2014}
\bibliography{tar2014} 

\end{document}

